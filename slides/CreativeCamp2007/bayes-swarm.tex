% presentation with pdflatex + foils package
% http://robotics.stanford.edu/~gerkey/tools/saynotopowerpoint.html

% use the foils document class, with large fonts and landscape orientation
\documentclass[20pt,landscape]{foils}

% setup the page geometry for landscape and use maximum screen real estate
\usepackage[pdftex]{geometry}
\geometry{headsep=2.0em,hscale=0.80}

% title, author, date
\title{Bayes-swarm\\bayesian web spidering}
\author{Matteo Zandi\\matteo.zandi@bayesfor.eu}
\date{6 ottobre 2007}

% the contents of \MyLogo are placed at the bottom center on the title slide
% and in the bottom left of other slides
\MyLogo{Matteo Zandi, Bayes-swarm}

% basic things that we need are below
\usepackage[italian]{babel}
\usepackage[utf8]{inputenc}
\usepackage{hyperref}
\hypersetup{
  pdftitle={Bayes-swarm, bayesian web spidering},
  pdfauthor={Matteo Zandi},
  pdfsubject={Bayes-swarm, research project which aims to spider web sources and extract data with bayesian models},
  pdfpagemode={FullScreen},
  pdfborder={0 0 0}
}
\usepackage{graphicx}

% set slide command
\newcommand{\slide}[1]{\foilhead{#1}}
%\newcommand{\slide}[1]{\foilhead{#1}\noindent}

% document
\begin{document}

\LogoOff
\maketitle
\begin{center}
  \includegraphics[width=0.3\hsize]{./figures/creative_camp_banner.png}
\end{center}

\slide{Cos'è Bayesfor?}
\LogoOn
\noindent
Una associazione che si propone di promuovere e realizzare ricerche, 
studi o sperimentazioni in materia di analisi dei dati e utilizzo di 
tecniche statistiche

\begin{center}
  http://bayesfor.eu
\end{center}

\slide{... e Bayes-swarm?}
\noindent
E' un progetto della associazione Bayesfor, che ha l'obiettivo di fare spidering 
di fonti sul web con lo scopo di estrarre informazioni come ad esempio:
\begin{itemize}
\item Correlazione tra parole nel tempo
\item Associazioni tra parole nelle fonti
\item Correlazione tra uso di parole e notizie
\item Correlazione tra uso di parole e mercati finanziari
\end{itemize}

\begin{center}
  http://bayes-swarm.googlecode.com
\end{center}

\slide{Come funziona}
\begin{itemize}
\item Lista di fonti (siti di quotidiani italiani ed esteri, agenzie
di stampa, etc)
\item Lista di parole "interessanti" (per ora, ma non per molto)
\item Ruby + gemme (ferret, etc)
\item Mysql, database in cui sono salvate tutte le informazioni
\end{itemize}

\slide{Alcuni numeri}
\begin{itemize}
\item 7 fonti (times, guardian, euronews, ny times, al jazeera, etc)
\item 22 pagine, circa 3 per fonte (pagina principale, economica, finanziaria, etc)
\item 87 parole interessanti (china, india, bush, iraq, terror, muslim, etc)
\item circa 1000 occorrenze a settimana
\end{itemize}

\slide{Grafico serie storica bush}
\begin{center}
  \includegraphics{./figures/timeseries_bush.pdf}
\end{center}
\noindent
Cosa è successo a metà agosto?

\slide{Grafico serie storiche china e india}
\begin{center}
  \includegraphics{./figures/timeseries_chinaindia.pdf}
\end{center}
\noindent
China e india tendono ad muoversi congiuntamente?

\slide{Grafico scatter di china e india}
\begin{center}
  \includegraphics{./figures/scatterplot_chinaindia.pdf}
\end{center}
\noindent
Coefficiente di correlazione $\cong 0.59$

\slide{Applicazione di modelli statistici 1/3}
\begin{center}
  \includegraphics{./figures/lm_hw_gp.pdf}
\end{center}
\noindent
Regressione semplice: $y=a+bx$.\\
Exponential smoothing: media di $n$ osservazioni precedenti con peso 
decrescente in funzione della distanza da oggi.\\
Processo gaussiano: i parametri variano nel tempo.

\slide{Applicazione di modelli statistici 2/3}
\begin{center}
  \includegraphics{./figures/bi_lm_gp.pdf}
\end{center}
\noindent
Minimi quadrati e processo gaussiano, caso bivariato

\slide{Applicazione di modelli statistici 3/3}
\begin{center}
  \includegraphics[width=0.4\hsize]{./figures/neural-prediction.png}
  \includegraphics[width=0.4\hsize]{./figures/neural-hist.png}
\end{center}
\noindent
Neural network

\slide{Visualizzazione con i grafi}
\begin{center}
  \includegraphics[width=0.4\hsize]{./figures/graphs.png}
\end{center}
\noindent
Ogni parola costituisce una 'stella', a parole con frequenza maggiore
sono associate stelle di dimensione maggiore. Ogni parola è connessa alle 
altre se sono state almeno una volta nella stessa pagina, la connessione 
è tanto più marcata quante più volte le parole si sono trovate nella stessa pagina.

\slide{Grazie per l'attenzione}
\noindent
\begin{center}
  http://bayesfor.eu\\http://bayes-swarm.googlecode.com
  
  \huge{domande?}
\end{center}

\end{document}
